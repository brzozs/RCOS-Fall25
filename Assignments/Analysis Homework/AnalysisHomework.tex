\documentclass[12pt]{article}
\usepackage{datetime}
\usepackage{color,array,graphics}
\usepackage{enumerate}
\usepackage[pdftex, colorlinks, linkcolor=red,citecolor=red,urlcolor=blue]{hyperref}
\usepackage{ulem}
\usepackage{listings}
\usepackage{pythonhighlight}
\usepackage{hyperref}

\newtheorem{theorem}{Theorem}
\newtheorem{claim}{Claim}
\newenvironment{solution}{\begin{proof}[Solution]}{\end{proof}}
\newtheorem{problem}{Problem}
\newtheorem{exercise}{Exercise}

\setlength{\parindent}{0cm}

\setlength{\parskip}{0.3cm plus4mm minus3mm}

\textwidth  6.5in
\oddsidemargin +0.0in
\evensidemargin +0.0in
\textheight 9.0in
\topmargin -0.5in

\usepackage{upquote,textcomp}
\usepackage{amssymb,amsmath,amsfonts,amsthm}
\usepackage{graphicx}
\usepackage{multicol}
\usepackage[T1]{fontenc}
\usepackage{xcolor}
 
\definecolor{codegreen}{rgb}{0,0.8,0.5}
\definecolor{codegray}{rgb}{0.5,0.5,0.5}
\definecolor{codepurple}{cmyk}{0, 0.7, 0.4, 0.3}
\definecolor{backcolour}{gray}{0.98}
\definecolor{textblue}{rgb}{.2,.2,.9}
\definecolor{textgreen}{rgb}{0,0.43,0}
 
\lstdefinestyle{mystyle}{
    backgroundcolor=\color{backcolour},   
    commentstyle=\color{textgreen},
    keywordstyle=\color{codepurple},
    numberstyle=\tiny\color{codegray},
    stringstyle=\color{codegreen},
    basicstyle=\ttfamily\small,
    breakatwhitespace=false,         
    breaklines=true,                 
    captionpos=b,                    
    keepspaces=true,                 
    numbers=left,                    
    numbersep=5pt,                  
    showspaces=false,                
    showstringspaces=false,
    showtabs=false,                  
    tabsize=2
}
 
\lstset{style=mystyle}
\def\OR{\vee}
\def\AND{\wedge}
\def\imp{\rightarrow}

\DeclareSymbolFont{AMSb}{U}{msb}{m}{n}
\DeclareMathSymbol{\N}{\mathbin}{AMSb}{"4E}
\DeclareMathSymbol{\Z}{\mathbin}{AMSb}{"5A}
\DeclareMathSymbol{\R}{\mathbin}{AMSb}{"52}
\DeclareMathSymbol{\Q}{\mathbin}{AMSb}{"51}
\DeclareMathSymbol{\I}{\mathbin}{AMSb}{"49}
\DeclareMathSymbol{\C}{\mathbin}{AMSb}{"43}

\usepackage{mathtools}
\DeclarePairedDelimiter\ceil{\lceil}{\rceil}
\DeclarePairedDelimiter\floor{\lfloor}{\rfloor}

\begin{document}
\begin{center}
\large
\textbf{CSCI 4470 Open Source Software}\\
\textbf{Analysis of an Open Source Project}\\
\textbf{Due date: Month/Day/Year}\\
\end{center}

In writing this analysis you may want to look at \textit{lab 5} report, particularly the end of the lab 
where we asked you to look at the license for 5 repositories on the RCOS page. For this assignment, 
we are going to ask you to take a more in depth look at some open source projects. \\

Pick three projects. You can use projects from \href{https://new.rcos.io}{RCOS}, 
\href{https://opensource.google/projects/}{Google}, \href{http://foss2serve.org/index.php/HFOSS_Projects}{foss2serve}, 
or they can be projects you find on your own. At least one of the projects must be outside of RCOS. 
You are going to write a report on the projects. The report should be in 2 parts. \\

For the first part, start by reading the \href{http://foss2serve.org/index.php/Project_Evaluation_(Activity)}{Project Analysis Activity}
on the foss2serve website. It walks you through an analysis of an open source project designed 
to characterize how suitable the project is for a specific class or area of interest. 
The instructions are specific to the OpenMRS project, but the analysis can be applied to any 
project. For each of the projects you choose, create a table based on the \href{http://foss2serve.org/index.php/Project_Evaluation_Rubric_(Activity)}{foss2serve rubric}. 
You must rank each project on every criteria and you must provide supporting evaluation data, 
but the data can be short. 1 sentence per criteria is fine. \\

Once this initial analysis is complete, pick the project that appeals to you the most and do a 
more in depth analysis. Expand on the table you generated: add more details to why you scored 
the project the way you did, discuss what the goal of the project is, discuss the technology 
used, and discuss the significance of the project. Why is this project appealing? What do you 
think interacting with this community would be like given the analysis you did? There is no firm 
length for this assignment, but I anticipate that a good report may be in the range of 3-5 pages. \\

\end{document}